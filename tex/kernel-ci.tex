\chapter{Linux kernel continuous integration}
\section{Context}

Presentation of Linux (what is it, role of OSS, community, subsystems, maintainers)
FE projects => upstreaming, product-tracking for clients, contribute to KernelCI project <= why CI
goals => custom tests, building a lab from scratch for not too much money

\section{State of the art}
unable to track all boards for all kernel versions
role of maintainers is to guarantee (or at least try to) the stability of old features and the working of new features => hard to track
answers from community is relatively slow = s.o. has to have the board to test the kernel (build the kernel with the same config etc...)
regressions are often detected on the go (hidden until someone uses the functionality)
unable to test features on all boards in the world
loss of time for rebuilding the kernel and setting the board each time someone needs help

\section{Project}

\section{Challenges encountered}
Bootloader updates (the process of updating or the impossibility to update => no bootz, no DHCP)
LAVA migration (v1->v2 no doc)
hardware problems (serial)

\section{Conclusion}
status of the tasks
=> no custom tests
=> miss half of the lab
=> some stats (number of boot tests)
=> project used by the engineers
