\chapter{Conclusion}

Les différents projets que j'ai réalisés au cours de mon stage effectué au sein de l'entreprise Free Electrons, ont porté sur plusieurs aspects.

Le sujet principal de mon stage était la création d'une ferme (ou laboratoire) de systèmes embarqués qui doit contribuer au projet KernelCI permettant d'apporter l'intégration continue au noyau Linux. L'objectif de ce projet est de tester le noyau Linux dans ses différentes configurations sur l'ensemble des systèmes embarqués mis à disposition par des laboratoires partenaires. Depuis ses plus jeunes années, Free Electrons est un fervent contributeur aux initiatives Open Source et particulièrement quand elles touchent leur c\oe{}ur de métier, à savoir le développement et la formation pour Linux. La réalisation de cette ferme en conjonction avec le projet KernelCI permet aux ingénieurs de l'entreprise de faire de la veille sur les produits des clients sans perdre le temps de recompiler et tester les différentes configurations du noyau Linux. C'est un gain de temps considérable en plus d'un suivi des produits de meilleure qualité. De plus, certains ingénieurs de Free Electrons sont également mainteneurs et profitent donc de ce projet pour s'aider dans leurs tâches. Enfin, c'est également une excellente opportunité d'aider la communauté du noyau Linux en mettant à disposition des cartes de développement d'anciens clients qui ne sont pas maintenues par nos ingénieurs. Le site du projet\footnote{\url{https://kernelci.org/}} est consulté régulièrement par la communauté des développeurs Linux et c'est une bonne vitrine pour l'entreprise de faire partie des 10 laboratoires partenaires du projet. Cependant, il reste une partie du sujet de stage à faire et qui pourrait bénéficier grandement aux ingénieurs de Free Electrons : implémenter une copie du projet KernelCI pour faire de l'intégration continue de nos noyaux Linux avec nos tests plus approfondis. Free Electrons va également recevoir de la visibilité parmi la communauté Open Source avec la publication de plusieurs articles ainsi qu'une présentation à l'Embedded Linux Conference Europe 2016 sur mon sujet de stage, présentant les différentes démarches et processus de création du laboratoire.

Parmi les sujets secondaires figurait le développement d'un outil permettant de prendre contrôle à distance les différents systèmes embarqués qui sont présents dans le laboratoire. Ce nouvel outil permet déjà depuis quelques mois aux ingénieurs de Free Electrons de travailler sur des systèmes embarqués auxquels ils ne peuvent pas accéder, que ce soit pour cause de participation à des conférences, à des formations ou de télétravail. Ce logiciel permet également de mieux partager le travail sur les différentes cartes de développement ajoutées au laboratoire. En effet, il est désormais beaucoup plus simple d'"échanger" une carte puisqu'il n'y a plus à la débrancher et la rebrancher, il suffit de se déconnecter de l'outil pour permettre à une autre personne de prendre le contrôle de cette même carte.\\
J'ai également ajouté le support d'une carte de développement à la fois dans le noyau Linux et le bootloader U-Boot pour un client qui souhaiterait faire une nouvelle version de son produit basé cette fois-ci sur la-dite carte de développement. Enfin, j'ai travaillé sur le développement d'un driver de Convertisseur Analogique-Numérique ou CAN qui sert à la fois de CAN, de contrôleur d'écran tactile et de sonde thermique du SoC. Avant que je ne développe ce driver, un autre driver existait déjà mais celui-ci ne permettait pas l'utilisation du CAN comme simple CAN, uniquement comme contrôleur d'écran tactile et sonde thermique.

Ce stage a été pour moi une première expérience très enrichissante dans le monde de l'informatique embarquée et a confirmé mon désir de travailler dans ce domaine. Il m'a permis de découvrir une bonne partie des différentes étapes de développement pour systèmes embarqués, allant du flashage du bootloader, de la compilation du noyau Linux, du développement de drivers et Device Tree à la création d'un root filesystem contenant les outils nécessaires au client. Une des plus grandes difficultés a été l'absence de documentation et quand il y en avait, de son incohérence avec d'autres.\\
Ce stage a également été cause de mes premières interactions avec les communautés Open Source et m'a permis de me rendre compte de l'utilité de partager son code pour les faire commenter, critiquer, discuter et valider par les développeurs de la communauté. Malgré une totale autonomie sur le sujet principal du stage qui a des fois été frustrante et difficile à gérer, la réussite des différents sujets ainsi que les multiples retours des différentes communautés et les nombreuses aides des ingénieurs de l'entreprise sur les autres sujets me permettent de tirer un bilan très positif de cette expérience à Free Electrons.
