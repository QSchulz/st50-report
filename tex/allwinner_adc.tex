\chapter{Driver for Allwinner SoCs' ADC}
\section{Context}

%FIXME: Correct touchscreen explanation
An Analog to Digital Converter or ADC is a device which converts a signal often representing a voltage or a current to a digital value. This component is really useful in data processing in embedded systems because most of them only understand discrete values. For example, a variable resistor such as a thermistor, a photoresistor, a variator or even a touchscreen controller. These resistors reduce current flow depending how they react to temperatures or light for example. The ADC works by comparing the input signal with a reference signal which is either ground or from another input (thus called differential ADC). The known function representing their behaviour can then be applied to the value returned by the ADC to get a "human-readable" value. An ADC can be used as a touchscreen controller because of the inner working of the touchscreen: the touchscreen is checkered with a lot of electrical lines and when a press occurs on the touchscreen, both a vertical and horizontal line have a slightly different voltage or current, therefore giving the position of the press.\\
In the Linux kernel, an input of an ADC is called a channel.

The Allwinner SoCs all have an ADC that can also act as a touchscreen controller and a thermal sensor. The first four channels can be used either for the ADC or the touchscreen and the fifth channel is used for the thermal sensor. However some ADCs only embed the thermal sensor.

There is currently a driver\footnote{\url{https://git.kernel.org/cgit/linux/kernel/git/torvalds/linux.git/tree/drivers/input/touchscreen/sun4i-ts.c}} which handles the ADC in touchscreen mode and the thermal sensor but there is no access to the ADC feature at all. This driver also only works for the SoCs which can act as an ADC and a touchscreen controller.

The goal is to write a generic driver for all Allwinner SoCs, even the one acting only as thermal sensor, which can handle all features of the ADC, be it the touchscreen controller, the ADC controller or the thermal sensor and replace the current driver.

\section{Project}

Allwinner A13, A20 and A31 ADCs have all features (ADC and touchscreen controllers and thermal sensor) whereas the A23 and A33 ones only have the temperature sensor. They all share the same registers addresses and behaviour except for the A31 which has a slightly different address for one of its register but they have different functions for computing the real value in Celsius of the thermal sensor. In the long term, the H3 could also use this driver but with some modifications since there are a lot of changes in registers addresses but its behaviour is globally the same as the aforementioned SoCs.

The ADC cannot operate at the same time as an ADC controller and as a touchscreen controller since the four channels of the Allwinner SoCs' ADC are used by either of them. Therefore, there must be an exclusion mechanism. Moreover, the thermal sensor is only returning valid values when the ADC operates as a touchscreen controller.\\
To switch between touchscreen and ADC modes, we need to poke a few bits in registers and (de)activate an interrupt which notifies when a press is released from the touchscreen.

Since a touchscreen may not be attached to the ADC as well as the ADC may only have one channel for the thermal sensor, the driver has to be modular, deactivating touchscreen or ADC controllers when they are not present physically in the board. The perfect system would be to have different drivers in different Linux kernel subsystems to handle the different parts of the ADC, one for the temperature, one for the ADC controller and one for the touchscreen controller. However, they share almost all the code and since they are part of the same component, they have to cooperate before communicating wih the it. This seems complex but there is one type of driver which handles this perfectly: the Multi Function Device, or MFD, driver in conjunction with regmap functions.

As presented in the previous chapter, drivers are probed if one of their compatibles is found in one of the Device Tree nodes. However, it would be tedious to edit the Device Tree each time we want a Linux kernel with or without a touchscreen controller for example. Fortunately, MFD drivers are capable of probing what are called subdrivers or MFD cells without using the Device Tree. The subdrivers are probed according to which compatible they match and are passed resources such as interrupts and properties as if they were probed from matching compatibles in a Device Tree.

Regmap functions are able to map memory and share it amongst different drivers which suits perfectly sharing memory between subdrivers of an MFD driver. It offers shared mechanisms for memory writing and reading as well as interrupt handling which are everything we need for our subdrivers.

The global principle is the following:

\begin{itemize}
  \item the MFD driver is probed thanks to compatible present in the Device Tree,
  \item the MFD driver maps the memory and interrupts,
  \item the MFD driver probes subdrivers with shared memory and interrupts;
\end{itemize}

The Allwinner SoCs' ADC driver handles most of the logic of all drivers because most of it is reading data from the ADC registers or switching modes. The other drivers, be it the touchscreen or the temperature driver, interact with the ADC driver for switching modes and request the ADC driver for data. The ADC driver has to explicitly share its channels with drivers and then and only then the other drivers will be able to request data from it. It is also possible to request data from the ADC driver from the system as a user by opening some special files which are handled by the driver and are basically triggers for data requests.

The touchscreen driver is expecting data structured as an array of touch coordinates (X and Y) to notify the input framework of the Linux kernel where a touch has occured. When the touchscreen is activated, it will trigger the mode switch in the ADC driver to change to touchscreen mode and ask the ADC driver to start to continuously read data from the hardware component. In the ADC driver, a buffer of touch coordinates representing the data in the hardware FIFO of the ADC is filled in with coordinates until the FIFO is emptied and then sent to the touchscreen driver which knows how to process the data. The latter driver also has to notify the input framework when the touch has ended, basically when the finger released all pression from the touchscreen. It does so by registering a handler for an hardware interrupt which occurs when there is an "up" event, understand a touch being released.\\
On the other side, there is already a driver named iio\_hwmon which exposes the temperature of the SoC by reading channels from different ADCs. Since our ADC driver exposes the channel used by the thermal sensor, we can use our ADC driver in conjunction with iio\_hwmon to let Linux users request the internal temperature of the SoC.

All hardware components consume energy when powered on but consume less when idle or even shutdown. Some of embedded systems like a phone, portable console or tablet are meant to be used with batteries and the less they consume, the longer they last. The ideal behaviour would be to shutdown the ADC when it is not used or when it hasn't been used for a while and power it on when someone or something needs to read data from it. The Linux kernel provides a mechanism for that called \textit{runtime\_pm} which does exactly as mentioned before by calling functions defined in each driver for resuming or suspending a driver and its related hardware components.

\section{Conclusion}
After a few weeks, I already had a working project with all the needed drivers but then I had to make my code \textit{mergeable} in the Linux kernel source code by proposing a patch to the Linux kernel community. This was my first intensive interaction with this community and it is still going on at the moment of the redaction. I learnt a lot from all the comments coming from different developers from different subsystems about code readability and factorisation as well as when to explicitly comment code and respect rules set up in each subsystem.

After four versions of my patches, I am confident of getting them validated by the community and thus being merged to the Linux source code. The patches for the touchscreen still need a lot of work with the community before it is determined to be acceptable.

Most of changes in the four versions were cosmetic but there were also few major mistakes such as NULL pointers dereferencing or functions used not as they should be. In addition, my patches require modifications in some frameworks and opened discussions on how we should do it. The difficult part is to separate the requirements for my patches to work and \textit{would-be-great-to-have} features in framework, which is basically learning how to say \textit{no} or \textit{later} to maintainers or developers.
