\chapter{Internship's mission}

\section{Presentation}
During my internship, I worked on the following subject: "Linux kernel continuous integration via KernelCI.org".

The \href{https://kernelci.org}{KernelCI.org} project's goal is to bring continuous integration to the Linux kernel. This involves testing the latest kernel versions on a wide range of systems to detect as soon as possible new regressions. This project is based on a community of labs, each lab having different systems to test.

The internship aims to build such a lab in Free Electrons' offices using the embedded systems the company already owns thanks to its work with some vendors, and to help solving detected problems from continuous integration's tests.

The internship schedule might be organized as following:
\begin{itemize}
  \item find how to boot the latest kernel version on each platform,
  \item implement a system of remote power control for each platform,
  \item add each platform to the platforms' pool of KernelCI,
  \item identify and try to solve the detected regressions,
  \item interact with the community to integrate the patches for these regressions in the bootloader or the kernel source code,
  \item create custom tests (e.g.: cryptography) for Free Electrons' systems;
\end{itemize}

In addition to the main subject, the internship might include work on small projects, i.e. the development of drivers for the Linux kernel or help on other projects depending on the needs of clients.

\section{Schedule}
The main subject should take three or four months of the internship total duration and the remaining would be split between drivers' development, contributions in the different projects and improvments of the tests.

I actually worked full-time on building the lab until May, 15th (i.e. 14 weeks) and then worked part-time on the same subject and on different projects.

Finally the internship's duration has been spread as following:
\begin{itemize}
  \item 17 weeks on building, maintaining, improving the lab (main subject),
  \item 9 weeks on side projects;
\end{itemize}
